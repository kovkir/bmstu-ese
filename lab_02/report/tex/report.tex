\documentclass{bmstu}

\begin{document}

\makereporttitle
{Информатика и системы управления (ИУ)}
{Программное обеспечение ЭВМ и информационные технологии (ИУ7)}
{2}
{Экономика программной инженерии}
{Определение ресурсов и затрат для проекта}
{}
{ИУ7-83Б}
{К.Э. Ковалец}
{М.Ю. Барышникова}


\setcounter{page}{2}

\section*{Тренировочное задание (вариант 0)}

\begin{enumerate}
    \item Дополнить временной план проекта, подготовленный на предыдущем этапе (лабораторная работа №1), информацией о ресурсах и определить стоимость проекта.
    \item Для этого заполнить ресурсный лист в программе MS Project, принимая во внимание, что к реализации проекта привлекается не более 10 человек.
    \item Предусмотреть, что стандартная ставка ресурса составляет 150руб./день.
    \item Произвести назначение ресурсов на задачи в соответствии с таблицей. С учетом того, что квалификация ресурсов одинаковая, при назначении ресурсов использовать процент загрузки.
    
    \imgs{table}{h!}{0.4}{Количество исполнителей для задач}

    \item Запланировать для выполнения работы Е использование материального ресурса, стоимостью 300 руб. за единицу и с нормой расхода 4 единицы в день.
\end{enumerate}

Был заполнен ресурсный лист (добавлены трудовые и материальные ресурсы).

\imgs{task0_1}{h!}{0.57}{Заполнение ресурсного листа}

\clearpage

Было произведено назначение ресурсов.

\imgs{task0_2}{h!}{1.2}{Пример назначения ресурсов}

\imgs{task0_3}{h!}{0.5}{Задачи проекта}

\imgs{task0_4}{h!}{0.6}{Диаграмма Ганта}

\clearpage

При выполнении задания возникли перегрузки.
Это связано с нехваткой ресурсов при одновременном выполнении работ.
В визуальном оптимизаторе ресурсов можно увидеть наложение задач исполнителей.

\imgs{task0_5}{h!}{0.65}{Наложение задач}

\section*{Содержание проекта}

Команда разработчиков из 16 человек занимается созданием карты города на основе собственного модуля отображения. Проект должен быть завершен в течение 6 месяцев. Бюджет проекта: 50 000 рублей.

\section*{Задание 1. Создание списка ресурсов}

В соответствии с заданием был заполнен ресурсный лист проекта.

\imgs{task1}{h!}{0.55}{Список ресурсов}

\clearpage

\section*{Задание 2. Назначение ресурсов задачам}

Все ресурсы были назначены задачам в соответствии с таблицей.

\imgs{task2_1}{h!}{0.6}{Диаграмма Ганта}

Ресурсы <<Системный аналитик>>, <<Художник-дизайнер>>, <<Технический писатель>> были одновременно задействованы в разных задачах, из-за чего возникли перегрузки.

\imgs{task2_2}{h!}{0.4}{Наложение задач}

Задачам 2, 8 и 12 было задано по 1000 р. фиксированных затрат.

\clearpage

Был добавлен новый трудовой ресурс <<Аренда сервера>>.
Стоимость аренды -- 2 рубля в час.

\imgs{task2_3}{h!}{0.55}{Добавление нового ресурса}

Для задачи №8 <<Построение базы объектов>> был арендован дополнительный сервер.

\imgs{task2_4}{h!}{1.2}{Назначение нового ресурса}

\imgs{task2_5}{h!}{0.55}{Состояние проекта после назначения ресурсов}

\clearpage

\section*{Задание 3. Анализ затрат по группам ресурсов}

Была проведена структуризация затрат по группам ресурсов.

\imgs{task3_1}{h!}{0.53}{Лист ресурсов}

После группировки данных по группам ресурсов были получены диаграммы информации о затратах и трудозатратах.

\imgs{task3_2}{h!}{0.75}{Информация о затратах}

\clearpage

\imgs{task3_3}{h!}{0.75}{Информация о трудозатратах}

\section*{Выводы}

На аренду сервера уходит значительная часть бюджета (13\%).
Группы <<Программирование>> и <<Ввод данных>> имеют схожие трудозатраты (29\% и 26\% соответственно). При этом затраты на <<Программированиие>> составили половину от всего бюджета проекта, в то время как затраты на <<Ввод данных>> составили всего 11\% бюджета. На груупу <<Анализ>> ушло сопоставимое кол-во затрат (10\%), однакоко трудозатраты составили всего 2\%, что в 13 раз меньше трудозатрат на <<Ввод данных>>.

При помощи программных средств была выявлена перегрузка определённых работников, т. к. они выполняют несколько задач одновременно. Требуется оптимизировать процесс планирования и распределения задач.

Затраты проекта составили 48 094 рубля, значит он уложился в бюджет, который составлял 50 000 рублей. Трудозатраты составили 9 377 ч.

\end{document}
