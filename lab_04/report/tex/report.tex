\documentclass{bmstu}

\begin{document}

\makereporttitle
{Информатика и системы управления (ИУ)}
{Программное обеспечение ЭВМ и информационные технологии (ИУ7)}
{4}
{Экономика программной инженерии}
{Актуализация параметров проекта. Ввод фактических данных для задач просмотр отклонений от контрольного плана}
{}
{ИУ7-83Б}
{К.Э. Ковалец}
{М.Ю. Барышникова}


\setcounter{page}{2}

\section*{Содержание проекта}

Команда разработчиков из 16 человек занимается созданием карты города на основе собственного модуля отображения. Проект должен быть завершен в течение 6 месяцев. Бюджет проекта: 50 000 рублей.

\section*{Сведения о проекте}

\imgs{task0_1}{h!}{0.52}{Краткая информация о ресурсах проекта}

\imgs{task0_2}{h!}{0.6}{Состояние проекта}

\section*{Индивидуальное задание}

Дата отчета: 21 апреля.

Проставить все работы по факту выполнения, кроме следующих:

\begin{itemize}
    \item 13 работа началась на неделю позже;
    \item 17 работа выполнена на 5\%;
    \item 6 работа закончилась на неделю позже.
\end{itemize}

Художник дизайнер с 15 марта уволился, нового нашли через неделю с увеличенной ставкой на 10\%.
Программист 3 заболел 28 марта на 10 дней.
Системный аналитик с 1 апреля на 40\% занят в другом проекте.
С 1 апреля на 5\% увеличилась аренда сервера.

\section*{Задание 1. Актуализация информации о сроках работ проекта}

Зададим дату отчета на 21 апреля.

\imgs{task1_1}{h!}{1}{Дата отчета}

Внесем фактические данные для отдельных задач проекта:

\begin{itemize}
    \item 13 работа началась на неделю позже (рисунок \ref{img:task1_2});
    \item 17 работа выполнена на 5\% (рисунок \ref{img:task1_3});
    \item 6 работа закончилась на неделю позже (рисунок \ref{img:task1_4}).
\end{itemize}

Состояние проекта после актуализация информации о сроках работ представлено на рисунке \ref{img:task1_5}.

\imgs{task1_2}{h!}{1}{13 работа началась на неделю позже}

\clearpage

\imgs{task1_3}{h!}{0.95}{17 работа выполнена на 5\%}

\imgs{task1_4}{h!}{0.95}{6 работа закончилась на неделю позже}

\imgs{task1_5}{h!}{0.4}{Состояние проекта после актуализация информации о сроках работ}

\clearpage

В результате бюджет проекта увеличился на 320 рублей (с 48 825 до 49 145 рублей), длительность проекта не изменилась. 

Также произошли перегрузки сотрудников в связи с тем, что совещания происходили во время их работы (рисунок \ref{img:task1_6}). Чтобы устранить перегрузки, было выполнено ручное выравнивание. Поиск превышений доступности был выставлен <<по месяцам>>.

\imgs{task1_6}{h!}{0.6}{Перегрузки сотрудников}

После устранения перегрузок бюджет и продолжительность проекта не изменились.

\section*{Задание 2. Учет непредвиденных событий}

Художник-дизайнер с 15 марта уволился, нового нашли через неделю с увеличенной ставкой на 10\%.

\imgs{task2_1}{h!}{0.8}{Доступность художник-дизайнера}

\clearpage

Так как в период с 15.03 по 21.03 осуществлялся поиск нового сотрудника на должность художник-дизайнера, то в этот период  в сведениях о ресурсе необходимо проставить 0\% в столбце единиц доступности ресурса. С 22.03 новый сотрудник устроился на работу, поэтому ресурс художник-дизайнера вновь стал доступным на 100\% (рисунок \ref{img:task2_1}).

\imgs{task2_2}{h!}{0.8}{Увеличение стандартной ставки художник-дизайнера}

Новый сотрудник устроился на работу с увеличенной ставкой на 10\%.
Пожтому необходимо с 22.03 увеличить стандартную ставку с 8.00 руб/ч до 8.80 руб/ч (рисунок \ref{img:task2_2}).

В результате бюджет проекта увеличился на 72 рубля (с 49 145 до 49 217 рублей), длительность проекта не изменилась.

\imgs{task2_3}{h!}{0.67}{Состояние проекта}

\clearpage

Программист 3 заболел 28 марта на 10 дней.

\imgs{task2_4}{h!}{0.8}{Доступность программиста 3}

После изменения доступности программсита 3 (рисунок \ref{img:task2_4}) произошли перегрузки ресурсов в связи с тем, что из-за болезни сотрудника некоторые задачи стали содержать ресурсы с превышением доступности (рисунок \ref{img:task2_5}). Чтобы устранить перегрузки, было выполнено ручное выравнивание. Поиск превышений доступности был выставлен <<по неделям>>.

\imgs{task2_5}{h!}{0.8}{Перегрузки ресурсов}

После изменения доступности программсита 3 и устранения перегрузок бюджет и продолжительность проекта не изменились.

\clearpage

Системный аналитик с 1 апреля на 40\% занят в другом проекте.

\imgs{task2_6}{h!}{0.72}{Доступность системного аналитика}

Так как с 1 апреля системный аналитик на 40\% будет занят в другом проекте, то его максимальная доступность в текущем проекте составит 60\% (рисунок \ref{img:task2_6}). После изменения доступности системного аналитика бюджет и продолжительность проекта не изменились.


С 1 апреля на 5\% увеличилась аренда сервера.

\imgs{task2_7}{h!}{0.72}{Увеличение стандартной ставки аренды сервера}

\clearpage

В результате бюджет проекта увеличился на 279 рубля (с 49 217 до 49 496 рублей), длительность проекта не изменилась (рисунок \ref{img:task2_8}).

\imgs{task2_8}{h!}{0.6}{Состояние проекта}

\clearpage

\section*{Выводы}

Для просмотра отклонений в проекте выведем линию прогресса. На рисунке \ref{img:task3_1} видно, что проект не отклонился от графика.

Из статистики проекта (рисунок \ref{img:task3_2}) можно увидеть, что после актуализации информации о сроках работ и учета всех непредвиденных событий проект продолжил укладывается в бюжет 50 000 рублей (бюджет увеличился с 48 825 до 49 496 рублей), а продолжительность работ и вовсе не изменилась (20.33 недели).

\imgs{task3_1}{h!}{0.4}{Линия прогресса на диаграмме Ганта}

\imgs{task3_2}{h!}{1.2}{Статистика проекта}

\end{document}
