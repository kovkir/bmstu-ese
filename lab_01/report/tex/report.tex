\documentclass{bmstu}

\begin{document}

\makereporttitle
{Информатика и системы управления (ИУ)}
{Программное обеспечение ЭВМ и информационные технологии (ИУ7)}
{1}
{Экономика программной инженерии}
{Планирование программного проекта в Microsoft Project}
{}
{ИУ7-83Б}
{К.Э. Ковалец}
{М.Ю. Барышникова}


\setcounter{page}{2}

\section*{Тренировочное задание (вариант 0)}

Осуществить планирование проекта со следующими временными характеристиками:

\imgs{table}{h!}{0.33}{Временные характеристики}

Дата начала проекта -- 1-й рабочий день февраля текущего года.
Провести планирование работ проекта, учитывая следующие связи между задачами:

\begin{enumerate}
    \item Предусмотреть, что A, E и F -- исходные работы проекта, которые можно начинать
    одновременно.
    \item Работы B и I начинаются сразу по окончании работы F.
    \item Работа J следует за E, а работа C -- за A.
    \item Работы H и D следуют за B, но не могут начаться, пока не завершена C.
    \item Работа G начинается после завершения H и J.
\end{enumerate}

% На рисунке \ref{img:task0} представлено решение тренировочного задания.

По умолчанию используется фиксированный объем ресурсов.

\imgs{task0}{h!}{0.54}{Решение тренировочного задания}

\clearpage

\section*{Содержание проекта}

Команда разработчиков из 16 человек занимается созданием карты города на основе собственного модуля отображения. Проект должен быть завершен в течение 6 месяцев. Бюджет проекта: 50 000 рублей.

\section*{Задание 1. Настройка рабочей среды проекта}

На вкладке \texttt{Проект -> Сведения} внесены параметры по условию.

\imgs{task1_1}{h!}{0.8}{Настройка сведений о проекте}

На вкладке \texttt{Файл -> Параметры -> Расписание} установлены параметры рабочей недели и планирования.

\imgs{task1_2}{h!}{0.7}{Настройка расписания}

\clearpage

На вкладке \texttt{Проект -> Изменить рабочее время} установлены нерабочие праздничные дни.

\imgs{task1_3}{h!}{0.65}{Настройка нерабочих праздничных дней}

На вкладке \texttt{Задача -> Суммарная задача} установлена суммарная задача проекта и добавлена заметка с основной информацией о проекте.

\imgs{task1_4}{h!}{0.65}{Настройка суммарной задачи}

\clearpage

\section*{Задание 2. Создание списка задач}

Введен список задач в соответствии с таблицей, представленной в задании лабораторной работы.

\imgs{task2}{h!}{0.4}{Список задач}

\section*{Задание 3. Структурирование списка задач}

При помощи кнопки \texttt{Понизить уровень задачи} были выделены подзадачи в соответствии с условием.

\imgs{task3}{h!}{0.4}{Разбиение на подзадачи}

\clearpage

\section*{Задание 4. Установление связей между задачами}

При помощи заполнения колонки \texttt{Предшественник} у каждой задачи были установлены связи между задачами.

\imgs{task4}{h!}{0.45}{Установленные связи между задачами}

\section*{Выводы}

В данной лабораторной работе были освоены возможности программы \texttt{Microsoft Project} для планирования проекта по разработке программного обеспечения. Был создан план проекта создания карты города. Была получена дата завершения работ -- 15.09.23. В итоге длительность проекта составила 6 месяцев и 15 дней.

\end{document}
