\documentclass{bmstu}

\begin{document}

\makereporttitle
{Информатика и системы управления (ИУ)}
{Программное обеспечение ЭВМ и информационные технологии (ИУ7)}
{5}
{Экономика программной инженерии}
{Контроль хода выполнения проекта с помощью средств анализа
затрат. Анализ рисков по методу PERT. Работа с отчетами}
{}
{ИУ7-83Б}
{К.Э. Ковалец}
{М.Ю. Барышникова}


\setcounter{page}{2}

\section*{Содержание проекта}

Команда разработчиков из 16 человек занимается созданием карты города на основе собственного модуля отображения. Проект должен быть завершен в течение 6 месяцев. Бюджет проекта: 50 000 рублей.

\section*{Индивидуальное задание}

Дата отчета: 21 апреля.

Проставить все работы по факту выполнения, кроме следующих:

\begin{itemize}
    \item 13 работа началась на неделю позже;
    \item 17 работа выполнена на 5\%;
    \item 6 работа закончилась на неделю позже.
\end{itemize}

Художник дизайнер с 15 марта уволился, нового нашли через неделю с увеличенной ставкой на 10\%.
Программист 3 заболел 28 марта на 10 дней.
Системный аналитик с 1 апреля на 40\% занят в другом проекте.
С 1 апреля на 5\% увеличилась аренда сервера.

\section*{Сведения о проекте}

\imgs{task0_1}{h!}{0.55}{Краткая информация о ресурсах проекта}

\clearpage

\imgs{task0_2}{h!}{0.36}{Состояние проекта}


\section*{Задание 1. Работа с таблицей освоенного объема}

Просмотрим таблицу освоенного объема (рисунок \ref{img:task1_1}). Для этого перейдем во вкладку <<Вид>> и выберем в ней <<Таблицы>> -> <<Другие таблицы>> -> <<Освоенный объем>>.

Данная таблица содержит следующие поля.

\begin{itemize}
    \item \textbf{Запланированный объем (3О)} --- это средства, которые были затра- чены на выполнение задачи в период с начала проекта до выбранной даты отчета, если бы задача точно соответствовала графику и смете.
    \item \textbf{Отклонение от календарного плана (ОКП)} --- позволяет вычислить несоответствие сметы, которое вызвано различием между плановым и фактическим объёмом работы, если это величина меньше 0, то проект опаздывает.
    \item \textbf{Отклонение по стоимости (ОПС)} --- сравнивает сметную и фактиче- скую стоимость выполненной работы и позволяет выделить несоответствие сметы, вызванные разницей стоимости ресурсов, если эта величина меньше нуля, то проект вышел за пределы сметы.
    \item \textbf{Предварительная оценка по завершении (ПОПЗ)} --- отображает ожидаемые общие затраты для задачи, расчет которых основан на предположении, что оставшаяся часть работы будет выполнена в точном соответствии со сметой (прогноз по завершении).
    \item \textbf{Затраты по базовому плану (БПЗ)} --- показывают фиксированные затраты и стоимость ресурсов согласно базовому плану.
    \item \textbf{Отклонение по завершению (ОПЗ)} --- разность между БПЗ и ПОПЗ, если эта величина отрицательна, то наблюдается перерасход средств.
\end{itemize}

\imgs{task1_1}{h!}{0.6}{Таблица освоенного объема}

Из полученной таблицы видно, что на дату отчетного периода освоенный объем (18 902 руб) больше запланированного (18 516 руб). Это связано с тем, что фактические траты на <<Создание интерфейса>> больше, чем запланированные. Также стоит заметить, что освоенный объем на <<Создание ядра GIS>> оказался меньше запланированного.

По таблице освоенного объема можно сделать следующие выводы.

\begin{itemize}
    \item ОКП > 0 --- означает, что проект идет с опережением.
    \item ОПС > 0 --- означает, что проект укладывается в смету.
    \item ОПЗ > 0 --- означает, что отсутствует перерасход средств.
\end{itemize}

\section*{Задание 2. Работа с отчетами проекта}

Создадим отчет о бюджетной стоимости (<<Отчеты>> -> <<Наглядные отчеты>> -> <<Отчет о бюджетной стоимости>>).

\imgs{task2_1}{h!}{0.8}{Создание отчета о бюджетной стоимости}

\clearpage

\imgs{task2_2}{h!}{0.8}{Отчет о бюджетной стоимости по кварталам}

Чтобы отобразить отчет по неделям, перейдем во вкладку <<Использование назначений>> (рисунок \ref{img:task2_3}) и раскроем кварталы.

\imgs{task2_3}{h!}{0.73}{Использование назначений}

\clearpage

\imgs{task2_4}{h!}{0.85}{Отчет о бюджетной стоимости по неделям}

Из рисунка \ref{img:task2_4} видно, что наибольшее количество денег понадобится на 5-ой неделе проекта. В это время выполнялись задачи <<Создание заставки>>, <<Анализ и построение структуры базы объектов>>, <<Анализ и проектирование ядра>>, <<Создание модели ядра>>.

Также проанализируем превышение затрат (<<Отчеты>> -> <<Затраты>> -> <<Превышение затрат>>).

\imgs{task2_5}{h!}{0.65}{Графики отклонения по стоимости}

На рисунке \ref{img:task2_5} видно превышение стоимости по задачам.

\begin{itemize}
    \item Превышение стоимости у художник-дизайнера, так как с 22 марта на эту должность наняли сотрудника с увеличенной ставкой на 10\%.
    \item Превышение стоимости у 3D аниматора, так как он работал над задачей <<Создание заставки>>, которая завершилась на неделю позже.
    \item Превышение стоимости аренды сервера, так как с 1 апреля на 5\% увеличилась стоимость ее аренды.
    \item Превышение стоимости у задачи <<Создание интерфейса>>, так как над ней работали художник-дизайнер и 3D аниматор, затраты на которых также увеличились.
    \item Превышение стоимости у задачи <<Построение базы объектов>>, так как в ней использовалась аренда сервера, стоимость которой с 1 апреля была увеличина.
    \item Превышение стоимости у задачи <<Создание руководства>>, так как над ней работал художник-дизайнер, затраты на которого также увеличились.
    \item Превышение стоимости у задачи <<Создание web-сайта и поддержка>>, так как над ней работали художник-дизайнер и 3D аниматор, затраты на которых также увеличились.
\end{itemize}

\section*{Задание 3. Декомпозиция проекта.}

Была проведена декомпозиция проекта на основе ЛР2. Проект был разбит на следующие этапы:

\begin{itemize}
    \item анализ;
    \item проектирование;
    \item разработка;
    \item тестирование;
    \item техническая поддержка.
\end{itemize}

\clearpage

На рисунке \ref{img:task3_1} представлено состояние проекта после выполнения ЛР2 с примененным ручным выравниванием по дням.

\imgs{task3_1}{h!}{0.53}{Состояние проекта до декомпозиции}

На рисунке \ref{img:task3_2} представлено состояние проекта после выполнения декомпозиции с примененным ручным выравниванием по месяцам.

\imgs{task3_2}{h!}{0.53}{Состояние проекта после декомпозиции}

\clearpage

На рисунке \ref{img:task3_3} представлено состояние проекта после выполнения декомпозиции с добавленными совещаниями и оптимизацией критического пути (на всех задачах связанных с программированием были задействованы все программисты).

\imgs{task3_3}{h!}{0.5}{Состояние проекта после декомпозиции с добавленными совещаниями и оптимизацией критического пути}

На рисунке \ref{img:task3_4} представлено состояние проекта после выполнения ЛР3.

\imgs{task3_4}{h!}{0.5}{Состояние проекта после выполнения ЛР3}

\clearpage

После проведения декомпозиции проекта из ЛР2, добавления совещаний, а также оптимизации критического пути удалось сократить затраты проекта (в сравнении с результатом ЛР3) на 5 930 рублей (с 48 825 до 42 895), но при этом возрасла длительность почти на месяц (дата окончания сдвинулась с 27.07 до 25.08). Так как был запас по времени, то после проведения декомпозиции проект остался в рамках заложенных 6 месяцев.

\section*{Выводы}

Из отчета о бюджетной стоимости видно, что наибольшее количество денег понадобится на 5-ой неделе проекта.

Декомпозиция проекта из ЛР2 показала свою эффективность, так как проект продолжил укладываться в заложенные 6 месяцев, но при этом затраты сократились на 5 930 рублей. 

Работа с отчетами позволяет проследить движение средств в проекте, вовремя перепланировать его в случае выхода за рамки сроков или бюджета.

\end{document}
